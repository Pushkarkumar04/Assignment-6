\documentclass{article}
\usepackage[utf8]{inputenc}
\usepackage{graphicx}

\title{ASSIGNMENT 6\\ON\\ Five Solutions To Covid19 Provided By Biomedical Engineers​}
\author{BY\\PUSHKAR KUMAR\\ROLL NO.: 21111040\\FIRST SEMESTER\\BRANCH:BIOMEDICAL ENGINEERING\\SECTION:A\\NATIONAL INSTITUTE OF TECHNOLOGY, RAIPUR\\ASSIGNMENT SUBMITTED TO\\
DEPARTMENT OF BIOMEDICAL ENGINEERING}
\date{}

\begin{document}

\maketitle
\begin{figure}[h]
    \centering
    \includegraphics[height=8cm,width=8cm]{download.jpg}
\end{figure}

\section{ Five Solutions To Covid19 Provided By Biomedical Engineers​
}

 Biomedical engineering bridges the gap between engineering and medicine, seamlessly combining the design and problem-solving skills of engineering with medical and biological sciences in order to improve healthcare diagnosis, monitoring and therapy.The COVID-19 pandemic has triggered the drive for technological advancement. 
 
 \subsection{Robotics}
 Robotic technologies have been utilized in many ways during the current pandemic: Robots were designed to communicate between patients and doctors at a distance (a); to disinfect surfaces with UV light (b) ; to deliver essential  medical supplies; to monitor vital signs (c) ; to remind people of infection prevention measures like social distancing; and to scale-up production of  diagnostic tools, drugs and vaccines.It’s heartwarming to hear that robots lighten the workload of health care providers and help them keep their distance from infected patients.  Delivery robots bring food and supplies to homes of at-risk people protecting them from delivery workers potentially infecting them. 
 
 \subsection{ Drones}
 Canada-based drone (a) tech firm Draganfly is rolling out stationary cameras and drones to detect people with COVID-19 symptoms. Drone engineers and enthusiasts use drones to monitor social distancing measures in a large gathering (b) and to deliver essential medical supplies to remote hospitals and clinics.  Virus spotting drones (c) were also used as early diagnostic tools by obtaining people’s temperatures and identifying the possible spread of infection.Engineers equipped drones with speakers to remind people of quarantine protocols. Even though drones are not considered medical devices based on their intended use, their role as support tools should be recognized in our health care system.
 \subsection{Vital signs monitor }
 It is mandatory to monitor COVID-19 patient vital signs during illness as well as during the recovery stage.  Vital sign monitors constantly check blood oxygen saturation, heart condition, breathing and the movement of the patient.  Since COVID-19 targets the lungs, it is obligatory to monitor the blood oxygen saturation (SpO2) also known as “peripheral capillary oxygen saturation”.  SpO2 refers to the percentage of red blood cells that are saturated with oxygen.As per interim guidance from World Health Organization (WHO), in a young adult severe pneumonia accompanied by high fever or a respiratory anomaly and a SpO2 level of less than or equal to 93percentage is one of the major symptoms of COVID-19.  A normal healthy person should be able to achieve SpO2 levels of 94percent to 99percent. For patients with mild respiratory diseases, the SpO2 should be 90percent or above.While in quarantine at home or recovering from the disease, wearable vital sign sensors provide peace of mind and assurance about a person’s health.  Some companies have launched wearable vital sensors, some are still in the development stage, and others are in clinical testing.  Examples include a wrist band for SpO2 monitoring (a), a sensor for monitoring breathing and body temperature (b) and a biometric sensor with location and position tracking (c).
 \subsection{ PPE}
 Medical practitioners must protect themselves from infection using personal protective equipment (PPE) such as face shield, gowns, gloves and masks.    However, due to the huge inflow of COVID-19 infected patients to the hospital and clinics, the supply of required PPE was initially insufficient. A number of medical personnel became infected and died after helping patients as a result of limited PPE, lack of proper training, and monitoring of use.Personal protective devices such as protective clothing was designed and made using surgical drapes and plastics (a).   Diagnostic testing booths equipped with HEPA filters (b) were made to reduce the use of   surgical gowns and medical supplies. This booth is more comfortable for healthcare personnel and lowers the probability of infection during the removal of gowns.
 \subsection{Telemedicine and remote monitor}
 In the absence of a vaccine, activities such as meeting people, gathering with a crowd and even visiting a doctor are a challenge. Innovation in communication using teleconferences or online consultations has increased significantly. Telemedicine was promoted several years ago, but gained little traction. However, during the COVID-19 pandemic, telemedicine now delivers medical care remotely to millions using communications technology(a). By using more sophisticated technologies like videoconferencing and other new emerging applications, telemedicine will continue to become popular in our health care system.
 
 
 
 
 




\end{document}

